\documentclass[doc, a4paper, apacite]{apa6}

\usepackage[american]{babel}

\usepackage{csquotes}
%\usepackage[style=apa,sortcites=true,sorting=nyt,backend=biber]{biblatex}
%\DeclareLanguageMapping{american}{american-apa}
\usepackage{threeparttable}
\usepackage{booktabs} % For nice tables
\usepackage{amsmath} % For \text{} function 
\usepackage{setspace}

\title{Decision-making and communication under uncertainty}
\shorttitle{DSTL07: Speaker to message}

\author{Charlotte E. R. Edmunds, Adam Harris, Magda Osman}
\affiliation{Queen Mary, UCL, University of London \\ 11 January 2021}

%\leftheader{Edmunds}

%\abstract{}
%
%\keywords{}

\begin{document}
	\doublespacing
	
Theoretical accounts of category learning and decision-making often assume that people make decisions based on a representation that integrates all the available information about a stimulus \cite{Pothos2011}.  
However, there is also experimental work that suggests that participants have a tendency to only use a subset of the available information \cite{Edmunds2020, Wills2015}. 
This observation raises an important practical question: are people aware of which pieces of information in an option are most important?
Another way of framing this is, are people sensitive to the degree of predictive uncertainty?
Predictive uncertainty can be thought of as a measure of the likelihood that an outcome follows a cue. 
For example, the likelihood that after eating rotting food (cue) you would get sick (outcome). 

In this experiment, we explore how sensitive participants are predictive uncertainty by exploring what they do in response to missing data. 
Further, we look to see whether the stimulus representation matters.
 Often, representations of entities in military circumstances often combine all the available information into a single glyph. 
By automatically integrating the information into a single holistic representation might support participants in combining information that they would otherwise find difficult and thus, attend more to less predictive dimensions. 


\section{Research questions}
Q1: By identifying the strategies participants use, we can determine whether participants integrate information from multiple sources to inform their decisions, or whether they base them on a subset of information. 
Q2: By varying the presentation of information, we can see whether using an integrated representation of the stimuli improves the likelihood that participants integrate information or not. In other words, are the participants more likely to use different strategies when the representation of information is different?
Q3: By testing participants in the absence of some of the information, we can see whether they are sensitive overall to the predictiveness of the dimensions. For instance, if we remove the most predictive dimension (90\%) then a sensitive participant should choose to categorise on the basis of the next predictive dimension (80\%). 
Q4: Finally, looking at the confidence judgements, we can see whether one method is better than the other in terms of accuracy or consistency. Thus, we can judge which communication format might be better for the “speaker” to document their subjective uncertainty.   

%\section{Method description given to DSTL}
%
%\subsection{Phase 1}
%Broadly, in this experiment, participants have to map entities (described by multiple pieces of AIS-like data) to either ``friendly’’ or ``hostile.’’ This experiment will have three stages: learning, test and report. 
%In the learning phase, on each trial participants will be shown the information concerning a particular entity. From that they will make a judgement as to whether it is friendly or hostile and then be given corrective feedback. We will provide five sources of information for each entity: the first source will be 90\% predictive, the second 80\%, the third 70\%, the fourth 60\% and the fifth 50\%. 
%These “information sources” will correspond to typical AIS data that might be used to make this judgements for actual vessels. For example, the most predictive dimension might be tonnage with 90\% of large vessels being hostile. 
%We plan to manipulate stimulus appearance between participants. In the separable representation, each dimension is displayed separately (perhaps as in Figure 1). This is similar to other ways that AIS data are commonly presented (see https://www.marinetraffic.com). In the integral representation, each dimension is displayed as part of a single glyph (as is common in the NATO symbology).
%By the end of this phase, we expect participants to have inferred a strategy that allows them to perform reasonably well at this categorisation task. We expect to see that the majority of participants use only the most predictive dimension and ignore the rest. 
%
%\subsection{Phase 2}
%In the second phase, we will present participants with the same types of stimuli. However, some of the dimensions will not be available to them. By removing dimensions (especially the more predictive dimensions) we can see whether participants gained any knowledge about the other dimensions. If a participant is sensitive to the predictive uncertainty of all the dimensions, they will select the second most predictive dimension, if they are not, they will likely select another dimension at random. Participants in this phase will have to again judge whether a particular vessel is hostile or friendly, and in addition they will have report their confidence in their judgements. We plan to manipulate how participants will give these confidence judgements: either as verbal descriptions ("Certain", "Likely", "Not likely", "Definitely not"), numerical judgements (the probability that the ship was whatever they said), or graphical (likely a pie chart).
%
%\subsection{Phase 3}
%Finally, participants will be asked to describe the strategy they used and why they chose that strategy in an open text box. This acts as a manipulation check for the modelling work that is required to determine participants’ strategies (Edmunds et al., 2018). 

\section{Method}
\subsection{Participants}
Data will be collected from participants online using Prolific Academic. 
Participants will be paid for their participation.

\subsection{Design}
This experiment has two between-subject conditions. 
Participants will either see the relevant category information as a text-based list or as an integrated glyph. 

\subsection{Procedure}
Currently, the screen display is black text on a white background. 
I think I will change this to something more reminiscent of MS-DOS. 

\subsubsection{1. Category learning}
First participants are welcomed to the trial and then shown the instructions. 
Then on each trial, they are shown a radar screen (with nothing on it) on the left of the screen and either the glyph or data table on the right. 
They must press either the f key for friendly or the h key for hostile. 
They receive feedback for 1000ms and an inter-trial interval of 500ms. 

In the category learning phase, participants are trained to a learning criterion.
This will probably be around 80\% but we need to double check that this is sensible. 

\subsubsection{2. Category test}
The category test phase is very similar to the category learning phase. 
However, participants are not given feedback. 
Further, the most predictive dimension is removed (and participants are warned about this in advance). 
In addition, participants in this phase will give confidence ratings. 

\subsubsection{3. Verbal reports}
In addition to the category learning and test phases, we will ask the participants to give feedback on how they completed the task. 
In order to encourage honesty, we will emphasise in the instructions that their answers will not affect their possible payment. 

They will be asked
\begin{enumerate}
	\item Which dimension of the five (list of dimensions) do you think was most useful in determining whether the ship was friendly or neutral?
	\item Which dimension(s) did you use the most?
	\item Why were those dimensions important?
	\item Imagine that another participant was asked to complete the experiment exactly as you did, how would you tell them to respond?
\end{enumerate}

These questions aim to determine a) what dimensions the participant thinks they were using, b) whether they had a coherent strategy and what it was and c) if they had an internal causal representation of the contingencies in the experiment. 

%\subsection{To change}
%\begin{itemize}
%	\item Display colours \\
%	\item Radar pictures - will probably need to sample from a bunch \\
%	\item Need to add more dimensions \\
%	\item Need to determine which dimensions are relevant \\
%	\item Double check the correlations between dimension-category structure mapping \\
%	\item Change instructions \\
%	\item Update stimuli with complete category structure \\
%	\item Update database so will collect additional data \\
%	\item Change keys to f for friendly and h for hostile (probably) \\
%	\item Think verbal report should have small box for the dimension names and large box for why \\
%\end{itemize}

\clearpage
\newpage
\bibliographystyle{apacite}
\bibliography{references}

\end{document}
