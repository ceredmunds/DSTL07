\documentclass[doc, a4paper, apacite]{apa6}

\usepackage[american]{babel}

\usepackage{csquotes}
%\usepackage[style=apa,sortcites=true,sorting=nyt,backend=biber]{biblatex}
%\DeclareLanguageMapping{american}{american-apa}
\usepackage{threeparttable}
\usepackage{booktabs} % For nice tables
\usepackage{amsmath} % For \text{} function 
\usepackage{setspace}

\title{Decision-making and communication under uncertainty}
\shorttitle{DSTL07: Speaker to message}

\author{Charlotte E. R. Edmunds, Adam Harris, Magda Osman}
\affiliation{Queen Mary, UCL, University of London \\ 11 January 2021}

%\leftheader{Edmunds}

%\abstract{}
%
%\keywords{}

\begin{document}
\maketitle
\doublespacing

In the literature review, we noted that most theoretical approaches in the categorisation and decision-making literatures argue that participants make decisions based on an integrated combinations of all the available information about a stimulus (Pothos \& Wills, 2011). However, experimental work has shown that participants have a tendency (or bias) to use only a subset of the available dimensions (Noguchi \& Stewart, 2018; Wills, Inkster, \& Milton, 2015). Evidence that appears to indicate information integration may rather be an artefact of participants relying on different subsets of the available information. This observation raised a question: are participants (or indeed Naval operators) aware of which pieces of information are most important for classifying stimuli (entities) as one of several types (hostile, friendly etc.)? In other words, are people sensitive to predictive uncertainty? Predictive uncertainty can be thought of as the likelihood that an outcome follows a cue. For example, the likelihood that after eating rotting food (cue) you would get sick (outcome). Thus, in this experiment, we explore how sensitive participants are predictive uncertainty by exploring what they do in response to missing data. 
Further, we look to see whether the stimulus representation matters. Often, representations of entities in military circumstances often combine all the available information into a single glyph. 
By automatically integrating the information into a single holistic representation might support participants in combining information that they would otherwise find difficult and thus, attend more to less predictive dimensions. 


\section{Research questions}
Q1: By identifying the strategies participants use, we can determine whether participants integrate information from multiple sources to inform their decisions, or whether they base them on a subset of information. 
Q2: By varying the presentation of information, we can see whether using an integrated representation of the stimuli improves the likelihood that participants integrate information or not. In other words, are the participants more likely to use different strategies when the representation of information is different?
Q3: By testing participants in the absence of some of the information, we can see whether they are sensitive overall to the predictiveness of the dimensions. For instance, if we remove the most predictive dimension (90\%) then a sensitive participant should choose to categorise on the basis of the next predictive dimension (80\%). 
Q4: Finally, looking at the confidence judgements, we can see whether one method is better than the other in terms of accuracy or consistency. Thus, we can judge which communication format might be better for the “speaker” to document their subjective uncertainty.   

\section{Method description given to DSTL}

\subsection{Phase 1}
Broadly, in this experiment, participants have to map entities (described by multiple pieces of AIS-like data) to either ``friendly’’ or ``hostile.’’ This experiment will have three stages: learning, test and report. 
In the learning phase, on each trial participants will be shown the information concerning a particular entity. From that they will make a judgement as to whether it is friendly or hostile and then be given corrective feedback. We will provide five sources of information for each entity: the first source will be 90\% predictive, the second 80\%, the third 70\%, the fourth 60\% and the fifth 50\%. 
These “information sources” will correspond to typical AIS data that might be used to make this judgements for actual vessels. For example, the most predictive dimension might be tonnage with 90\% of large vessels being hostile. 
We plan to manipulate stimulus appearance between participants. In the separable representation, each dimension is displayed separately (perhaps as in Figure 1). This is similar to other ways that AIS data are commonly presented (see https://www.marinetraffic.com). In the integral representation, each dimension is displayed as part of a single glyph (as is common in the NATO symbology).
By the end of this phase, we expect participants to have inferred a strategy that allows them to perform reasonably well at this categorisation task. We expect to see that the majority of participants use only the most predictive dimension and ignore the rest. 

\subsection{Phase 2}
In the second phase, we will present participants with the same types of stimuli. However, some of the dimensions will not be available to them. By removing dimensions (especially the more predictive dimensions) we can see whether participants gained any knowledge about the other dimensions. If a participant is sensitive to the predictive uncertainty of all the dimensions, they will select the second most predictive dimension, if they are not, they will likely select another dimension at random. Participants in this phase will have to again judge whether a particular vessel is hostile or friendly, and in addition they will have report their confidence in their judgements. We plan to manipulate how participants will give these confidence judgements: either as verbal descriptions ("Certain", "Likely", "Not likely", "Definitely not"), numerical judgements (the probability that the ship was whatever they said), or graphical (likely a pie chart).

\subsection{Phase 3}
Finally, participants will be asked to describe the strategy they used and why they chose that strategy in an open text box. This acts as a manipulation check for the modelling work that is required to determine participants’ strategies (Edmunds et al., 2018). 

\section{Current method}

\subsection{Procedure}
Currently, the screen display is black text on a white background.
\subsubsection{1. Category learning}
First participants are welcomed to the trial and then shown incorrect instructions. 
Then on each trial, they are shown a radar screen (with nothing on it) on the left of the screen and two dimensions with data on the right. 
They must press either the f or j keys to say whether it is in category f or j. 
They receive feedback for 1000ms and an ITI of 500ms. 

\subsubsection{3. Verbal reports}
They are asked:
\begin{enumerate}
	\item Please describe which stimulus dimension you thought was most important \\
\end{enumerate}

\subsection{To change}
\begin{itemize}
	\item Display colours \\
	\item Radar pictures - will probably need to sample from a bunch \\
	\item Need to add more dimensions \\
	\item Need to determine which dimensions are relevant \\
	\item Double check the correlations between dimension-category structure mapping \\
	\item Change instructions \\
	\item Update stimuli with complete category structure \\
	\item Update database so will collect additional data \\
	\item Change keys to f for friendly and h for hostile (probably) \\
	\item Think verbal report should have small box for the dimension names and large box for why \\
\end{itemize}

\section{To ask Magda and Adam}
\begin{itemize}
	\item What verbal report questions do you think are important? \\
	\item Which dimensions do you think we should use? \\
	\item What ideas do you have about formatting/appearance? \\
	\item 
\end{itemize}


\clearpage
\newpage
\bibliographystyle{apacite}
\bibliography{references}

\end{document}
