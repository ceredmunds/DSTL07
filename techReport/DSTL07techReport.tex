\documentclass[doc, a4paper, apacite]{apa6}

\usepackage[american]{babel}

\usepackage{csquotes}
%\usepackage[style=apa,sortcites=true,sorting=nyt,backend=biber]{biblatex}
%\DeclareLanguageMapping{american}{american-apa}
\usepackage{threeparttable}
\usepackage{booktabs} % For nice tables
\usepackage{amsmath} % For \text{} function 
\usepackage{setspace}

\title{Decision-making and communication under uncertainty}
\shorttitle{DSTL07: Sensitivity to uncertainty}

\author{Charlotte E. R. Edmunds, Andy J. Wills, Adam Harris, Magda Osman}
\affiliation{Queen Mary, UCL, University of London \\ 11 January 2021}

%\leftheader{Edmunds}

%\abstract{}
%
%\keywords{}

\begin{document}
\maketitle
	\doublespacing
	
\section{Method}

\subsection{Participants}
Participants were recruited through Prolific Academic.
We limited the sample to participants between the ages of 18 and 65 with corrected to normal vision. 
Participants were paid a flat fee of \pounds 2 for completing the learning phase.
Those that passed the learning phase were given a bonus of \pounds 6 if they then completed the following two test phases. 

This sample size was based on an \emph{a priori} statistical power analysis using G*Power 3.1 \cite{GPower2007, GPower2009}. 
We assumed a medium effect size (partial $\eta^2 = .06$) with $\alpha=0.05$ and power $(1-\beta)= 0.95$.
The projected total sample size was $N=206$.
We rounded this up to $N=208$ so to have even numbers of participants in each experimental group. 

\subsection{Category structure}
The abstract category structure that all participants learned is shown in Table~\ref{table:abstractStructure}. 
The stimuli are constructed from five binary dimensions. 
In the experiment, Category X and Y will be randomly assigned to the labels `Friendly' and `Hostile.' 
In addition, the dimensions will be randomly assigned to the label (and therefore perceptual features) of Craft (airplane, submarine), Speed (fast, slow), Direction (left, right), Type (autonomous, decoy) and Status (fully operational, damaged). 

\begin{table}[t]
	\centering
	\caption{}
	\label{table:abstractStructure}
	\begin{tabular}{lllllllllll}
		\toprule
		\multicolumn{5}{c}{Category X} &  & \multicolumn{5}{c}{Category Y} \\
		\cline{1-5} \cline{7-11} \\
		D1   & D2   & D3   & D4  & D5  &  & D1   & D2   & D3   & D4  & D5  \\
		\midrule
		1    & 1    & 1    & 1   & 1   &  & 0    & 0    & 0    & 0   & 0   \\
		1    & 1    & 1    & 1   & 0   &  & 0    & 0    & 0    & 0   & 1   \\
		1    & 1    & 1    & 0   & 1   &  & 0    & 0    & 0    & 1   & 0   \\
		1    & 1    & 1    & 0   & 0   &  & 0    & 0    & 0    & 1   & 1   \\
		1    & 1    & 0    & 1   & 1   &  & 0    & 0    & 1    & 0   & 0   \\
		1    & 1    & 0    & 1   & 0   &  & 0    & 0    & 1    & 0   & 1   \\
		1    & 1    & 0    & 0   & 1   &  & 0    & 0    & 1    & 1   & 0   \\
		1    & 0    & 1    & 1   & 0   &  & 0    & 1    & 0    & 0   & 1   \\
		1    & 0    & 1    & 0   & 1   &  & 0    & 1    & 0    & 1   & 0   \\
		0    & 1    & 1    & 1   & 0   &  & 1    & 0    & 0    & 0   & 1   \\
		\bottomrule
	\end{tabular}	
\end{table}

Thus, participants could score 100\% accuracy if they used a ``holistic'' or ``overall similarity'' judgment: stimuli with 3 or more 1's predict Category 1, and stimuli with 3 or more 0's predict Category 0. 
Alternatively, using a single dimension rule based on the first dimension they could score 90\%, or using the second dimension 80\%. 
A final high performing possibility is a rule-plus-exception strategy.
Participants learn that the first dimension predicts category membership, apart from two exceptions. 
This would again score them 100\%. 
However, the pattern of generalisation would likely be different in the test phase where the primary dimension was removed. 

\begin{figure}[b]
	\centering
	\includegraphics[width=0.33\textwidth]{images/integratedStimulusExample}
	\includegraphics[width=0.33\textwidth]{images/integratedStimulusExample2}
	\caption{Example of stimulus dimensions as integrated stimuli.}
	\label{fig:integratedStimuli}
\end{figure}

\subsection{Design}
The experiment had a 2 (spatial presentation: separated, integrated) x 2 (social: operator, superior) between-subjects design. 

Participants were randomly assigned to see the stimuli as either spatially integrated (as in Figure~\ref{fig:integratedStimuli}) or spatially separated (as in Figure~\ref{fig:separatedStimulus}). 
Additionally, they were either told that their confidence ratings are to be given to their commander or to one of their fellow operators. 

\begin{figure}[t]
	\centering	
	\includegraphics[width=0.6\textwidth]{images/separatedStimulusExample}
	\caption{An example stimulus presented as a separated set of symbols. (Incorrect: speed was given as a number as in the integrated stimulus diagram.)}
	\label{fig:separatedStimulus}	
\end{figure}

\subsection{Procedure}
At the beginning of the experiment, participants were welcomed and told where to write to if they had any queries or concerns. 
The experiment had three blocks of trials: a learning phase and two types of test phase. 

\subsubsection{1. Category learning} 
Participants were told the overall structure of the experiment and told what each of the perceptual features of the glyph represent. 
They were then given specific instructions for the category learning phase. 
On each trial they were shown a stimulus (either separated or integrated depending on the condition they were assigned).
They then had to click one of two buttons labelled `Friendly' or `Hostile' to make their response.
Then they were shown the feedback `Correct!' or `Wrong', as appropriate, for $1000ms$.
This was followed by a blank screen for $500ms$.

The stimuli in this section were pseudo-randomised so every 20 trials the participants saw all the training stimuli in a random order. 
Participants were trained to a learning criterion of 80\% correct in the last 20 trials. 
Thus, the fewest number of trials a participant could see were 20 trials. 
Participants who did not exceed the learning criterion in 200 trials were excluded from the rest of the experiment. 

\subsubsection{2. Test with all dimensions}
After a short, self-paced break, participants moved to the second phase. 
Here, participants were tested on all 32 stimuli created from the binary levels of 5 stimulus dimensions. 
In this phase, there were 5 rounds, where participants saw all 32 stimuli in a random order. 
On every trial, they saw a stimulus and had to click on one of two buttons to indicate whether they thought it was friendly or hostile. 
No feedback was provided. 
On every 4 trials, starting from the first trial, participants were also asked to rate their confidence in the category judgment they had just given. 
Participants rated their confidence using a slider that went from `No idea' to `Certain.'
The data from this scale was discretised into 100 levels. 
There was an inter-trial interval of 500ms. 

Following this, there was an opportunity for participants to inform us how they completed the task. 
We reminded them of what the levels of each stimulus dimension meant, then we asked them to rank the five stimulus dimensions based on how important they were for classification. 
Then, we provided an open text box and asked participants to describe how exactly they decided whether a craft was friendly or hostile in the previous task.

\subsubsection{3. Test removing a dimension}
The final test phase was identical to the previous one in all but one respect: the stimuli shown to the participant only had four dimensions. 
In this final test phase, we removed the dimension that was 90\% predictive for that participant. 
For participants in the integrated condition, it was simply absent from the display. 
For participants in the separated condition, the box was filled with the text ``Information not available.''

\section{Results}

\subsection{Learning phase}
\subsubsection{Trials to criterion}
Here, we analyse the number of trials it took for participants to reach the learning criterion as shown in Figure~\ref{fig:DSTL07trialsCriterion}. 
For each condition, we removed outlying participants, i.e. those who took more trials than the upper quartile plus 1.5 times the interquartile range. 
This resulted in removing 3 participants from the integrated-operator condition, 2 from the integrated-superior condition and 7 from the separated-operator condition. 

Levene's test showed homogeneity of variance was violated, $F(3,195)=7.98$, $p<.001$.
Similarly, a Shapiro-Wilks test showed the data violated assumptions of normality, $W=0.89$, $p<.001$. 
Therefore, to correct for the violation of the normality assumption, we report an ANOVA based on the log transformed number of trials to criterion and its Bayesian equivalent. 
The ANOVA found a hint of an interaction between display condition and social condition, $BF=1.39905$, $F(1, 195)=4.33$, $p=.039$. 
There was a greater difference between the display conditions in the superior conditions, $M_\text{integrated}=3.64$, $SE=0.09$, $M_\text{separated}=3.89$, $SE=0.08$, than in the operator conditions, $M_\text{integrated}=3.51$, $SE=0.09$, $M_\text{separated}=3.40$, $SE=0.09$. 
The interaction was moderated by a main effect of social condition, $BF=53.66$, $F(1, 195)=12.74$, $p<.001$. 
Participants took longer to learn in the operator condition, $M=3.45$, $SE=0.06$, than in the superior condition, $M=3.76$, $SE=0.06$. 
The effect of display condition did not reach significance, $BF=0.24$, $F(1, 195)=1.054$, $p=.306$. 

\begin{figure}
	\centering
	\includegraphics{images/DSTL07trialsCriterion}
	\caption{Mean number of trials to reach the criterion for participants in each condition. Error bars are difference-adjusted confidence intervals.}
	\label{fig:DSTL07trialsCriterion}
\end{figure}

\subsubsection{Reaction times}
Here, we analyse the reaction time during the learning phase of the experiment as shown in Figure~\ref{fig:DSTL07learningRT}. 
For each condition, we removed outlying participants, i.e. those who took more trials than the upper quartile plus 1.5 times the interquartile range. 
This resulted in removing 2 participants from the integrated-operator condition, 5 from the integrated-superior condition, 3 from the separated-operator condition and 2 from the separated-superior condition. 

Levene's test showed that the assumption of homogeneity of variance was violated, $F(3, 195)=3.80$, $p=.011$. 
Similarly, a Shapiro-Wilks test showed that the data violated assumptions of normality, $W=0.94$, $p<.001$. 
Therefore, to correct for the violation of nomality, we report an ANOVA based on the log transformed reaction time. 
There was a small effect of display condition, $BF=2.98$, $F(1, 195)=12.27$, $p<.001$.
Participants were slower to respond in the separated condition, $M=8.27$, $SE=0.05$, than in the integrated condition, $M=8.01$, $SE=0.05$. 
There was also a slight difference in social condition, $BF=0.59$, $F(1, 195)=4.068$, $p=.045$. 
Participants were slower to respond in the superior condition, $M=8.21$, $SE=0.05$, than in the operator condition, $M=8.06$, $SE=0.05$. 
There was no evidence of an interaction, $BF=0.27$, $F(1, 195)=2.78$, $p=.097$. 

\begin{figure}
	\centering
	\includegraphics{images/DSTL07learningRT}
	\caption{Mean reaction time in milliseconds for participants in each condition. Error bars are difference-adjusted confidence intervals.}
	\label{fig:DSTL07learningRT}
\end{figure}

\subsection{Test phases}

\subsubsection{Descriptives}
Figures~\ref{fig:DSTL07testAccuracy} and \ref{fig:DSTL08testRT} respectively show the accuracy and reaction times of participants across the two types of test phases.
As some novel stimuli were added in both test phases, these graphs reflect the accuracy and speed only on the trials for which the participants had been trained (i.e. that they had seen before). 
Generally, we can see that accuracy performance drops considerably, suggesting that most participants were not able to identify the second most predictive dimension for those stimuli. 
However, the effects of display condition and social condition did not seem to make an impact. 

\begin{figure}
	\centering
	\includegraphics{images/DSTL07testSummaryAccuracy}
	\caption{Mean accuracy for participants in each condition in the test phase (with all attributes) and the partial test phase (where the most predictive attribute was removed). Error bars are difference adjusted confidence intervals.}
	\label{fig:DSTL07testAccuracy}	
\end{figure}

For reaction times, they may be a hint of a suggestion that decision making in the separated condition took longer. 
Further, it seems as if participants were slightly slower if they believed they were reporting to a superior compared to an operator. 

\begin{figure}
	\centering
	\includegraphics{images/DSTL07testSummaryRT.pdf}	
	\caption{Mean reaction time in milliseconds for participants in each condition in the test phase (with all attributes) and the partial test phase (where the most predictive attribute was removed.) Error bars are difference adjusted confidence intervals.}
	\label{fig:DSTL08testRT}
\end{figure}

\subsubsection{Analysis}
To analyse the test phase data, we conducted a mixed ANOVA with display condition and social condition as between-subject factors and experiment phase as a within-subject factor. 
First, we consider accuracy as the dependent variable. 
As participants have only been trained on 20 of the 32 stimuli in the test phases, we limit our analysis to the training stimuli. 
Here, the only effect that reached significance was the main effect of experiment phase, $BF=6.85 \times 10^{23}$, $F(1, 410)=33.64)$, $p<.001$. 
Unsurprisingly, participants performed better in the test phase with all dimensions than the test phase where the most predictive dimension was removed. 

\begin{table}
	\centering
	\caption{Analysis of accuracy in test phases}
	\label{table:analyisAccuracyTest}
	\begin{tabular}{l c c c}
		\toprule
		Factor & Bayes factor & F-value & p-value \\
		\midrule
		Display & 0.15 & 0.43 & .513 \\
		Social & 0.15 & 0.25 & .621 \\
		Display*Social & 0.41 & 2.98 & .085 \\
		Display*Phase & 0.16 & 0.08 & .783 \\
		Social*Phase & 0.16 & 0.64 & .426 \\
		Display*Social*Phase & 0.28 & 0.93 & .335 \\
		\bottomrule
	\end{tabular}
\end{table}

We now consider reaction time as the dependent variable. 
Again, there was a main effect of experiment phase on reaction time, $BF=6.14 \times 106{5}$, $F(1, 410)=7.47$, $p=.006$. 
Participants were faster when all of the information was available than when the most predictive dimension was removed. 

\begin{table}
	\centering
	\caption{Analysis of reaction time in test phases.}
	\label{table:analyisRTTest}
	\begin{tabular}{l c c c}
		\toprule
		Factor & Bayes factor & F-value & p-value \\
		\midrule
		Display & 0.53 & 4.82 & .029 \\
		Social & 0.33 & 0.81 & .370 \\
		Display*Social & 0.26 & 0.93 & .334 \\
		Display*Phase & 0.24 & 1.11 & .293 \\
		Social*Phase & 0.15 & 0.01 & .923\\
		Display*Social*Phase & 0.22 & 0.10 & .748 \\
		\bottomrule
	\end{tabular}
\end{table}

\subsubsection{Strategy analysis}

When looking at the strategies used in the second test phase, 4 participants (one in each condition) chose to deal with the uncertainty by simply pressing the same button throughout the entire phase.
In other words, they pressed ``Hostile'' say for every single trial.
All these participants were found to be using the most predictive dimension in the first test phase. 
Looking at the verbal reports, one participant (integrated-operator) seemed to struggle with learning the task, their comments being along the lines of ``I don't know, I was never told what was right.''
This, despite the fact they passed the learning criterion. 
The other three participants commented that they could not categorise 
These participants are excluded from the following analyses. 

In Tables~\ref{table:strategiesTest1}, \ref{table:strategiesTest2} and \ref{table:strategiesTest2withoutDim2}, we report the best fitting strategy by each participant. 
Here, we consider unidimensional rules based on dimensions 1, 2 etc. 
Additionally, we consider an overall similarity strategy where, for instance, a craft would be Category 1 if the stimulus had 3 or more 1's. 
For the partial test phase, this meant that we had to exclude some trials as they were not diagnostic of either category using an overall similarity sort. 
In the meantime, remember that when the below talks about the ``best'' model/strategy, is it the best \emph{of the ones fitted}.

So, in Table~\ref{table:strategiesTest1}, we can see that participants did quite well in classifying. 
Most appeared to use one of the two most predictive dimensions (dimensions 1 and 2, at 90\% and 80\% predictive). 
Perhaps interesting is that there isn't a linear relationship between predictability and how many participants chose it. 
There are more people using dimension 1 than dimension 2, but after that it gets messy. 
Perhaps people cannot distinguish between 60\% and 70\% predictive? 
It would be interesting to look to see whether participants that ended up best fit by less predictive dimensions took longer to learn. 

\begin{table}
	\centering
	\caption{Strategies identified in Test Phase 1, where all dimensions were shown.}
	\label{table:strategiesTest1}
	\begin{tabular}{rllrrrrrr}
	  \hline
	 & displayCondition & socialCondition & 1 & 2 & 3 & 4 & 5 & OS\\ 
	  \hline
      1 & integrated & operator &  24 &  15 &   2 &   7 &   4 &   1 \\ 
	  2 & integrated & superior &  33 &   9 &   3 &   4 &   1 &   2 \\ 
	  3 & separated & operator &  29 &  16 &   4 &   1 & - &   2 \\ 
	  4 & separated & superior &  23 &  14 &   5 &   9 &   1 &   1 \\ 
	   \hline
	\end{tabular}
\end{table}

In Table~\ref{table:strategiesTest2}, we can see the best fitting strategies for participants in the second test phase. 
This looks as if the majority of participants were sensitive to the the underlying contingencies of more than one dimension as most participants used dimension 2 in this phase. 

\begin{table}[b]
	\centering
	\caption{Strategies identified in Test Phase 2, where only dimensions 2-5 were shown to participants}
	\label{table:strategiesTest2}
	\begin{tabular}{rllrrrrrr}
	  \hline
	 & displayCondition & socialCondition & 1 & 2 & 3 & 4 & 5 & OS\\ 
	  \hline
	  1 & integrated & operator & - &  19 &   8 &  13 &   9 &   4 \\ 
	  2 & integrated & superior &   2 &  17 &   8 &  14 &   6 &   5 \\ 
	  3 & separated & operator & - &  23 &  12 &   9 &   2 &   6 \\ 
	  4 & separated & superior &   1 &  23 &  11 &  10 &   6 &   2 \\ 
	   \hline
	\end{tabular}
\end{table}

However, thinking more carefully, this is not the case. 
In Table~\ref{table:strategiesTest2withoutDim2}, I removed the participants who were identified as using dimension 2 in the first test phase. 
These participants would have no reason to switch from their unidimensional rule given that it was shown in both phases. 
Once these participants are removed we can see that participants are reasonably evenly dispersed across the remaining four dimensions. 
This suggests that participants are not sensitive to the underlying predictiveness of the dimensions. 
Further, this doesn't seem to vary depending on either the display or social conditions. 

\begin{table}
	\caption{Strategies identified in Test Phase 2, where only dimensions 2-5 were shown to participants. Additionally, here, participants who used the second dimension in Test Phase 1 were excluded.}
	\label{table:strategiesTest2withoutDim2}
	\centering
	\begin{tabular}{rllrrrrrr}
	  \hline
	 & displayCondition & socialCondition & 1 & 2 & 3 & 4 & 5 & OS\\ 
	  \hline
	  1 & integrated & operator &  &   9 &   7 &  11 &   8 &   3 \\ 
	  2 & integrated & superior &   2 &   9 &   8 &  14 &   6 &   4 \\ 
	  3 & separated & operator &  &  10 &  12 &   9 &   1 &   4 \\ 
	  4 & separated & superior &   1 &  10 &  10 &  10 &   6 &   2 \\ 
	   \hline
	\end{tabular}
\end{table}

\subsection{Confidence}
In analysing the confidence ratings, the main question we wished to answer was whether participants' confidence was influenced by knowledge of who was using the ratings. 
Additionally, we expected participants to be less certain when the most predictive dimension had been removed compared to when they had all the information available. 
So, we conducted a 2 (display condition) by 2 (social condition) by 2 (test phase) ANOVA. 
As expected there was a main effect of experiment phase, $BF=1.34 \times 10^{42}$, $F(1, 2102)=211.30$, $p<.001$. 
Participants were more confident when all of the information was present, $M=68.2$, $SE=0.69$, than when the most predictive dimension was removed, $M=54.1$, $SE=0.69$. 
However, none of the other main effects or interactions reached significance. 
Please see Table~\ref{table:confidenceAnalysis}. 
Note that the display condition almost reached significance in the ANOVA. 
However, the Bayesian analysis still suggests a Bayes factor under a third and so little weight should be placed on this effect. 

\begin{table}[b]
	\centering
	\caption{Analysis of confidence ratings}
	\label{table:confidenceAnalysis}
	\begin{tabular}{l c c c}
		\toprule
		Effect & Bayes factor & F value & p value \\
		\midrule
		Display & 0.25 & 3.589 & 0.058 \\
		Social & 0.12 & 1.91 & 0.167 \\
		Experiment phase $\times$ social & 0.08 & 0.85 & 0.356 \\
		Experiment phase $\times$ display & 0.12 & 1.80 & 0.180 \\
		Social $\times$ display & 0.07 & 0.64 & 0.424 \\
		Three-way interaction & 0.05 & 0.05 & 0.824 \\
		\bottomrule
	\end{tabular}
\end{table}

\subsection{Verbal reports}
\subsubsection{Dimension rankings}
Participants were asked to rank the importance of each dimension in determining whether it was friendly or hostile at the end of each test phase. 
To analyse that, we looked at the Spearman's rank correlation between the dimension and the participants rankings. 
After the full test phase, the Spearman's correlation across all conditions did not reach significance, $\rho=-0.003$, $S=187444097$, $p=.930$.
Similarly, after the partial test phase, the correlation across all conditions did not reach significance, $\rho=0.03$, $S=178147831$, $p=.378$. 
In Table~\ref{table:spearmanCorrelations}, we report the correlations by condition. 

\begin{table}
	\centering
	\caption{Spearman correlations between dimensions and rankings by condition.}
	\label{table:spearmanCorrelations}
	\begin{tabular}{l c c c c c}
	\toprule
	Test phase & Display condition & Social condition & $\rho$ & S & $p$-value \\
	\midrule
	Phase 2 & integrated & operator & 0.03 & 2855266 & .685 \\
	& integrated & superior & 0.07 & 2728527 & .271 \\
	& separated & operator & -0.06 & 2920812 & .365 \\
	& separated & superior & -0.05 & 3220007 & .418 \\
	\\
	Phase 3 & integrated & operator & 0.03 & 2805882 & .620 \\
	& integrated & superior & 0.02 & 2639406 & .727 \\
	& separated & operator & -0.04 & 2863690 & .565 \\
	& separated & superior & -0.05 & 3220007 & .418 \\
	\bottomrule
	\end{tabular}
\end{table}


\clearpage
\newpage
\bibliographystyle{apacite}
\bibliography{references}

\end{document}
